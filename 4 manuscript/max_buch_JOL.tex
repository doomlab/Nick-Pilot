\documentclass[english,,man]{apa6}
\usepackage{lmodern}
\usepackage{amssymb,amsmath}
\usepackage{ifxetex,ifluatex}
\usepackage{fixltx2e} % provides \textsubscript
\ifnum 0\ifxetex 1\fi\ifluatex 1\fi=0 % if pdftex
  \usepackage[T1]{fontenc}
  \usepackage[utf8]{inputenc}
\else % if luatex or xelatex
  \ifxetex
    \usepackage{mathspec}
  \else
    \usepackage{fontspec}
  \fi
  \defaultfontfeatures{Ligatures=TeX,Scale=MatchLowercase}
\fi
% use upquote if available, for straight quotes in verbatim environments
\IfFileExists{upquote.sty}{\usepackage{upquote}}{}
% use microtype if available
\IfFileExists{microtype.sty}{%
\usepackage{microtype}
\UseMicrotypeSet[protrusion]{basicmath} % disable protrusion for tt fonts
}{}
\usepackage{hyperref}
\hypersetup{unicode=true,
            pdftitle={Investigating the Interaction between Associative, Semantic, and Thematic Database Norms for Memory Judgments and Retrieval},
            pdfauthor={Nicholas P. Maxwell~\& Erin M. Buchanan},
            pdfkeywords={judgments, memory, association, semantics, thematics},
            pdfborder={0 0 0},
            breaklinks=true}
\urlstyle{same}  % don't use monospace font for urls
\ifnum 0\ifxetex 1\fi\ifluatex 1\fi=0 % if pdftex
  \usepackage[shorthands=off,main=english]{babel}
\else
  \usepackage{polyglossia}
  \setmainlanguage[]{english}
\fi
\usepackage{graphicx,grffile}
\makeatletter
\def\maxwidth{\ifdim\Gin@nat@width>\linewidth\linewidth\else\Gin@nat@width\fi}
\def\maxheight{\ifdim\Gin@nat@height>\textheight\textheight\else\Gin@nat@height\fi}
\makeatother
% Scale images if necessary, so that they will not overflow the page
% margins by default, and it is still possible to overwrite the defaults
% using explicit options in \includegraphics[width, height, ...]{}
\setkeys{Gin}{width=\maxwidth,height=\maxheight,keepaspectratio}
\IfFileExists{parskip.sty}{%
\usepackage{parskip}
}{% else
\setlength{\parindent}{0pt}
\setlength{\parskip}{6pt plus 2pt minus 1pt}
}
\setlength{\emergencystretch}{3em}  % prevent overfull lines
\providecommand{\tightlist}{%
  \setlength{\itemsep}{0pt}\setlength{\parskip}{0pt}}
\setcounter{secnumdepth}{0}
% Redefines (sub)paragraphs to behave more like sections
\ifx\paragraph\undefined\else
\let\oldparagraph\paragraph
\renewcommand{\paragraph}[1]{\oldparagraph{#1}\mbox{}}
\fi
\ifx\subparagraph\undefined\else
\let\oldsubparagraph\subparagraph
\renewcommand{\subparagraph}[1]{\oldsubparagraph{#1}\mbox{}}
\fi

%%% Use protect on footnotes to avoid problems with footnotes in titles
\let\rmarkdownfootnote\footnote%
\def\footnote{\protect\rmarkdownfootnote}


  \title{Investigating the Interaction between Associative, Semantic, and
Thematic Database Norms for Memory Judgments and Retrieval}
    \author{Nicholas P. Maxwell\textsuperscript{1,2}~\& Erin M.
Buchanan\textsuperscript{1}}
    \date{}
  
\shorttitle{Judgments and Recall}
\affiliation{
\vspace{0.5cm}
\textsuperscript{1} Missouri State University\\\textsuperscript{2} University of Southern Mississippi}
\keywords{judgments, memory, association, semantics, thematics}
\usepackage{csquotes}
\usepackage{upgreek}
\captionsetup{font=singlespacing,justification=justified}

\usepackage{longtable}
\usepackage{lscape}
\usepackage{multirow}
\usepackage{tabularx}
\usepackage[flushleft]{threeparttable}
\usepackage{threeparttablex}

\newenvironment{lltable}{\begin{landscape}\begin{center}\begin{ThreePartTable}}{\end{ThreePartTable}\end{center}\end{landscape}}

\makeatletter
\newcommand\LastLTentrywidth{1em}
\newlength\longtablewidth
\setlength{\longtablewidth}{1in}
\newcommand{\getlongtablewidth}{\begingroup \ifcsname LT@\roman{LT@tables}\endcsname \global\longtablewidth=0pt \renewcommand{\LT@entry}[2]{\global\advance\longtablewidth by ##2\relax\gdef\LastLTentrywidth{##2}}\@nameuse{LT@\roman{LT@tables}} \fi \endgroup}


\DeclareDelayedFloatFlavor{ThreePartTable}{table}
\DeclareDelayedFloatFlavor{lltable}{table}
\DeclareDelayedFloatFlavor*{longtable}{table}
\makeatletter
\renewcommand{\efloat@iwrite}[1]{\immediate\expandafter\protected@write\csname efloat@post#1\endcsname{}}
\makeatother
\usepackage{lineno}

\linenumbers

\authornote{Nicholas P. Maxwell received his masters degree at
Missouri State University and is now a Ph.D.~candidate at the University
of Southern Mississippi. Erin M. Buchanan is an Associate Professor of
Psychology at Missouri State University. Compliance with Ethical
Standards: The authors declare that they have no conflict of interest.

Correspondence concerning this article should be addressed to Nicholas
P. Maxwell, 901 S. National Ave, Springfield, MO, 65897. E-mail:
\href{mailto:nicholas.maxwell@usm.edu}{\nolinkurl{nicholas.maxwell@usm.edu}}}

\abstract{
This study examined the interactive relationship between semantic,
thematic, and associative word pair strength in the prediction of item
judgments and cued-recall performance. Participants were recruited from
Amazon's Mechanical Turk and were given word pairs of varying
relatedness to judge for their semantic, thematic, and associative
strength. After completing a distractor task, participants then
completed a cued recall task. First, we sought to expand previous work
on judgments of associative memory (JAM) to include semantic and
thematic based judgments, while also replicating bias and sensitivity
findings. Next, we tested for an interaction between the three database
norms (FSG, COS, and LSA) when predicting participant judgments and also
extended previous work to test for interactions between the three
database norms when predicting recall. Significant three-way
interactions were found between FSG, COS, and LSA when predicting
judgments and recall. For low semantic feature overlap, thematic and
associative strength were competitive; as thematic strength increased,
associative predictiveness decreased. However, this trend reversed for
high semantic feature overlap, wherein thematic and associative strength
were complementary as both set of simple slopes increased together.
Overall, our findings indicate the degree to which the processing of
associative, semantic, and thematic information impacts cognitive
processes such as retrieval and item judgments, while also examining the
underlying, interactive relationship that exists between these three
types of information.


}

\usepackage{amsthm}
\newtheorem{theorem}{Theorem}[section]
\newtheorem{lemma}{Lemma}[section]
\theoremstyle{definition}
\newtheorem{definition}{Definition}[section]
\newtheorem{corollary}{Corollary}[section]
\newtheorem{proposition}{Proposition}[section]
\theoremstyle{definition}
\newtheorem{example}{Example}[section]
\theoremstyle{definition}
\newtheorem{exercise}{Exercise}[section]
\theoremstyle{remark}
\newtheorem*{remark}{Remark}
\newtheorem*{solution}{Solution}
\begin{document}
\maketitle

The study of cognition has a rich history of exploring the role of
association in human memory. One key finding is that elements of
cognitive processing play a critical role in how well an individual
retains learned information. Throughout the mid-20th century,
researchers investigated this notion, particularly through the use of
paired-associate learning (PAL). In this paradigm, participants are
presented with a pair of items and are asked to make connections between
them so that the presentation of one item (the cue) will in turn trigger
the recall of the other (the target). Early studies of this nature
focused primarily on the effects of meaning and imagery on recall
performance. For example, Smythe and Paivio (1968) found that noun
imagery played a crucial role in PAL performance; subjects were much
more likely to remember word-pairs that were low in meaning similarity
if imagery between the two was high. Subsequent studies in this area
focused on the effects of mediating variables on PAL tasks as well as
the effects of imagery and meaningfulness on associative learning
(Richardson, 1998), with modern studies shifting their focus towards a
broad range of applied topics such as how PAL is affected by aging
(Hertzog, Kidder, Powell-Moman, \& Dunlosky, 2002), its impacts on
second language acquisition (Chow, 2014), and even in evolutionary
psychology (Schwartz \& Brothers, 2013).

Early PAL studies routinely relied on stimuli generated from word lists
that focused extensively on measures of word frequency, concreteness,
meaningfulness, and imagery (Paivio, 1969). However, the word pairs in
these lists were typically created due to their apparent relatedness or
frequency of occurrence in text. While lab self-generation appears face
valid, one finds that this method of selection lacks a decisive method
of defining the underlying relationships between the pairs (Buchanan,
2010), as these variables only capture psycholinguistic measurements of
an individual concept (i.e., how concrete is \emph{cat} and word
occurrence). PAL is, by definition, used on word pairs, which requires
examining concept relations in a reliable manner. As a result, free
association norms have become a common means of indexing associative
strength between word pairs.

As we will use several related variables, it is important to first
define association as the context-based relation between concepts,
usually found in text or popular culture (Nelson, McEvoy, \& Dennis,
2000). Such word associations typically arise through their
co-occurrence together in language. For example, the terms \emph{peanut}
and \emph{butter} have become associated over time through their joint
use to depict a particular type of food, though separately, the two
concepts share very little in terms of meaning. To generate these norms,
participants engage in a free association task, in which they are
presented with a cue word and are asked to list the first related target
word that comes to mind. The probability of producing a given response
to a particular cue word, or forward strength, can then be determined by
dividing the number of participants who produced the response in
question by the total number of responses generated for that word (FSG;
Nelson et al., 2000). Using this technique, researchers have developed
databases of associative word norms that can be used to generate stimuli
with a high degree of reliability. Many of these databases are now
readily available online, with the largest one consisting of over 72,000
associates generated from more than 5,000 cue words (Nelson, McEvoy, \&
Schreiber, 2004). More recently, the Small World of Words project (SWOW;
De Deyne, Navarro, \& Storms, 2013) has sought to capture associations
between Dutch words by employing a multiple response technique in
contrast to the traditional single response free association task used
by Nelson et al. (2004). These norms are now being collected for English
words (De Deyne, Navarro, Perfors, Brysbaert, \& Storms, 2018).

Similar to association norms, semantic word norms provide researchers
with another option of constructing stimuli for use in word-pair tasks.
These norms measure the underlying concepts represented by words and
allow researchers to tap into aspects of semantic memory. Semantic
memory is best described as an organized collection of our general
knowledge and contains information regarding a concept's meaning
(Hutchison, 2003). Models of semantic memory broadly fall into one of
two categories. Connectionist models (e.g., Rogers \& McClelland, 2006;
Rumelhart, McClelland, \& PDP Research Group, 1986) portray semantic
memory as a system of interconnected units representing concepts, which
are linked together by weighted connections representing knowledge. By
triggering the input units, activation will then spread throughout the
system activating or suppressing connected units based on the weighted
strength of the corresponding unit connections (Jones, Willits, \&
Dennis, 2015). On the other hand, distributional models of semantic
memory posit that semantic representations are created through the
co-occurrences of words together in a body of text and suggest that
words with similar meanings will appear together in similar contexts
(Riordan \& Jones, 2011). Popular distributional models of semantic
memory include Latent Semantic Analysis (LSA; Landauer \& Dumais, 1997)
and the Hyperspace Analogue to Language (HAL; Lund \& Burgess, 1996).

Feature production tasks are a common means of producing semantic word
norms (Buchanan, Holmes, Teasley, \& Hutchison, 2013; McRae, Cree,
Seidenberg, \& McNorgan, 2005; Vinson \& Vigliocco, 2008) In such tasks,
participants are shown the name of a concept and are asked to list what
they believe the concept's most important features to be (McRae et al.,
2005). Several statistical measures have been developed which measure
the degree of feature overlap between concepts. Similarity between any
two concepts can be measured by representing them as vectors and
calculating the cosine value (COS) between them (Maki, McKinley, \&
Thompson, 2004). Cosine values range from 0 (unrelated) to 1 (perfectly
related). For example, the pair \emph{hornet} - \emph{wasp} has a COS of
.88, indicating a high degree of overlap between the two concepts.
Feature overlap can also be measured by JCN, which involves calculating
the information content value of each concept and the lowest
super-ordinate shared by each concept using an online dictionary, such
as WordNET (Miller, 1995). The JCN value is then computed by summing
together the difference of each concept and its lowest super-ordinate
(Jiang \& Conrath, 1997; Maki et al., 2004). The advantage to using COS
values over JCN values is the limitation of JCN being tied to a somewhat
static dictionary database, while a semantic feature production task can
be used on any concept to calculate COS values. However, JCN values are
less time consuming to obtain if both concepts are in the database
(Buchanan et al., 2013).

Semantic relations can be broadly described as being taxonomic or
thematic in nature. Whereas taxonomic relationships focus on the
connections between features and concepts within categories (e.g.,
\emph{bird} - \emph{pidgeon}), thematic relationships center around the
links between concepts and an overarching theme or scenario (e.g.,
\emph{bird} - \emph{nest}; Jones \& Golonka, 2012). Jouravlev and McRae
(2016) provide a list of 100 thematic relatedness production norms,
which were generated through a task similar to feature production in
which participants were presented with a concept and were asked to list
names of other concepts they believed to be related. Distributional
models of semantic memory also lend themselves well to the study of
thematic word relations. Because these models are text-based and score
word pair relations in regard to their overall context within a
document, they assess thematic knowledge as well as semantic knowledge.
Additionally, text-based models such as LSA are able to account for both
the effects of context and similarity of meaning, bridging the gap
between associations and semantics (Landauer, Foltz, Laham, Folt, \&
Laham, 1998).

Discussion of these measures then leads to the question of whether each
one truly assesses some unique concept or if they simply tap into our
overall linguistic knowledge. Taken at face value, word pair
associations and semantic word relations appear to be vastly different,
yet the line between semantics/associations and thematics is much more
blurred. While thematic word relations are indeed an aspect of semantic
memory and include word co-occurrence as an integral part of their
creation, themes also appear to be indicative of a separate area of
linguistic processing. Previous research by Maki and Buchanan (2008)
appears to confirm this theory. Using clustering and factor analysis
techniques, they analyzed multiple associative, semantic, and text-based
measures of associative and semantic knowledge. First, their findings
suggested associative measures to be separate from semantic measures.
Additionally, semantic information derived from lexical measures (e.g.,
COS, JCN) was found to be separate from measures generated from analysis
of text corpora, suggesting that text-based measures may be more
representative of thematic information.

While it is apparent that these word relation measures are assessing
different domains of our linguistic knowledge, care must be taken when
building experimental stimuli through the use of normed databases, as
many word pairs overlap on multiple types of measurements. For example,
some of the first studies on semantic priming used association word
norms for stimuli creation (Lucas, 2000; Meyer \& Schvaneveldt, 1971;
Meyer, Schvaneveldt, \& Ruddy, 1975). This observation becomes
strikingly apparent when one desires the creation of word pairs related
on only one dimension. One particular difficulty faced by researchers
comes when attempting to separate association strength from feature
overlap, as highly associated items tend to be semantically related as
well. Additionally, a lack of association strength between two items may
not necessarily be indicative of a total lack of association, as
traditional norming tasks typically do not produce a large enough set of
responses to capture all available associations between items. Some
items with extremely weak associations may inevitably slip through the
cracks (Hutchison, 2003). As such, the present study seeks to provide
further insight by examining how different levels of associative overlap
(measured in FSG), semantic overlap (feature overlap measured with COS),
and thematic overlap (measured with LSA) affect cognitive tasks such as
short term item retrieval and item relatedness judgments. Instead of
focusing solely on one variable or trying to create stimuli that
represent only one form of relatedness, we included a range of each of
these variables to explore their potential interaction.

Specifically, this research was conceptualized within the framework of a
three-tiered view of the interconnections between these systems as it
relates to processing concept information. The three-tiered view is
inspired by models of reading and naming, particularly the triangle
models presented by Seidenberg and McClelland (1989) and Plaut (1995).
These models explored the nature of reading as bidirectional relations
between semantics, orthography, and phonology. In this research, we
examine if the associative, semantic, and thematic systems are
interactive for judgment and recall processes, much like the proposed
interactive nature of phonology, orthographics, and semantics for
reading and naming processes. Potentially, association, semantic, and
thematic facets of word relation each provide a unique contribution that
can be judged and used for memory, thus, suggesting three separate
networks of independent information. This view seems unlikely, in that
research indicates that there is often overlap in the information
provided by each measure of word-pair relatedness. Instead, dynamic
attractor networks, as proposed by Hopfield (1982) and McLeod, Shallice,
and Plaut (2000) may better represent the interplay between these
representations of concepts, as these models posit a similar feedback
relationship between concepts in a network. Using these models as a
theoretical framework for our study, we sought to understand how these
three types of word-pair information may interact when judgment and
recall processes were applied to concept networks, and use it as a
framework for exploring how associative, semantic, and thematic memory
networks share interconnections. Therefore, this study provides evidence
of the structure and interplay between different forms of network
relations for two cognitive tasks of judgment and retrieval and will
shed light on the underlying processing for each task.

\#\#Application to Judgment Studies

Traditional judgment of learning tasks (JOL) can be viewed as an
application of the PAL paradigm; participants are given pairs of items
and are asked to judge how accurately they would be able to correctly
respond with the target with the cue on a recall task. Judgments are
typically made out of 100, with a participant response of 100 indicating
full confidence in recall ability. In their 2005 study, Koriat and Bjork
examined overconfidence in JOLs by manipulating associative relations
(FSG) between word-pairs and found that subjects were more likely to
overestimate recall for pairs with little or no associative relatedness.
Additionally, this study found that when accounting for associative
direction, subjects were more likely to overestimate recall for pairs
that were high in backwards strength but low in forward strength. To
account for this finding, the authors suggested that JOLs may rely more
heavily on overlap between cue and target with the direction of the
associative relationship being secondary. Take for example the pair
\emph{feather} - \emph{bird}, which has a FSG of .051 and a BSG of .359.
This item pair also has a cosine value of .272 (suggesting low to
moderate feature overlap) and an LSA score of .517 (suggesting moderate
thematic overlap). As such, some of the overconfidence in JOLs may be
attributed more than just item associations. Paired items may also be
connected by similar themes or share certain features, resulting in
inflated JOLs.

Expanding upon this research, the traditional judgment of learning task
(JOL) can be manipulated to investigate perceptions of word pair
relationships by having participants judge how related they believe the
cue and target items to be (Maki, 2007a, 2007b). The judged values
generated from this task can then be compared to the normed databases to
create a similar accuracy function or correlation as is created in JOL
studies. When presented with the item pair, participants are asked to
estimate the number of people out of 100 who would provide the target
word when shown only the cue (Maki, 2007b), which mimics how the
association word norms are created through free association tasks. Maki
(2007a) investigated such judgments within the context of associative
memory by having participants rate how much associative overlap was
shared between items and found that responses greatly overestimated the
actual overlap strength for pairs that were weak associates, while
underestimating strong associates; thus replicating the Koriat and Bjork
(2005) findings for relatedness judgments based upon associative memory,
rather than judgments based on learning.

The judgment of associative memory function (JAM) is created by plotting
the judged values by the word pair's normed associative strength and
calculating a fit line, which characteristically has a high intercept
(bias) with a shallow slope (sensitivity). Figure \ref{fig:makislope}
illustrates this function. Overall, the JAM function has been found to
be highly reliable and generalized well across multiple variations of
the study, with item characteristics such as word frequency, cue set
size (QSS), and semantic similarity all having a minimal influence on it
(Maki, 2007b). Furthermore, an applied meta-analysis of more than ten
studies on JAM indicated that bias and sensitivity are nearly
unchangeable, often hovering around 40-60 points for the intercept and
.20-.30 for the slope (Valentine \& Buchanan, 2013). Additionally,
Valentine and Buchanan (2013) extended this research to include
judgments of semantic memory with the same results.

\begin{figure}
\centering
\includegraphics{max_buch_JOL_files/figure-latex/makislope-1.pdf}
\caption{\label{fig:makislope}JAM slope findings from Maki (2007a). JAM is
characterized by a high intercept (between 40 and 60) and a shallow
slope (between 0.20 and 0.40). The solid line shows expected results if
judgment ratings are perfectly calibrated with association norms.}
\end{figure}

The present study combined the paradigms of PAL, JOLs, and JAM to
examine item recall and judgments for three types of judgments of
relatedness (JORs) to explore the underlying memory network that is used
for each of these cognitive processes as described above. We tested the
following hypotheses based on previous research and semantic memory
models:

\begin{enumerate}
\def\labelenumi{\arabic{enumi})}
\item
  First, we sought to expand previous Maki (2007b), Maki (2007a),
  Buchanan (2010), and Valentine and Buchanan (2013) research to include
  three types of JORs in one experiment, while replicating JAM bias and
  sensitivity findings. We used the three database norms for
  association, semantics, and thematics to predict each type of JOR and
  calculated average slope and intercept values for each participant.
  First, we expected to find slope and intercept values that were
  significantly different from zero. Though the three types of word
  relations are distinct from one another, we should expect to find
  slopes and intercepts for semantic and thematic JORs to be within the
  range of previous JAM findings if these memory systems are
  interconnected. Finally, we examined the frequency of each predictor
  being the strongest variable to predict its own judgment condition
  (i.e., how often association was the strongest predictor of
  associative JORs, etc.). This hypothesis explores if judgment findings
  replicate across a range of variables and covariates (rather than each
  individually, as previous JOL and JAM publications) and expands our
  knowledge on how the judgment process taps into the underlying memory
  network.
\item
  Next, we explored the predictions from semantic network models that
  the relation between association, semantics, and thematics would be
  bidirectional in nature (i.e., the three-tiered hypothesis of each
  type of knowledge stacked in memory). Therefore, we expected to find
  an interaction between database norms in predicting JORs. We used
  multilevel modeling to examine the interaction of database norms for
  association, semantics, and thematics in relation to participant
  judgments.
\item
  These analyses were then extended to recall as the dependent variable
  of interest. We tested for the interaction of database norms in
  predicting recall by using a multilevel logistic regression, while
  controlling for judgment condition and rating. We expected to find
  that database norms would show differences in recall based on the
  levels of other variables (the interaction would be significant), and
  that ratings would also positively predict recall (i.e., words that
  participants thought were more related would be remembered better).
  Because judgment and recall are different cognitive processes, we used
  this hypothesis to examine how memory networks may be differently
  interactive for memory in comparison to judgment.
\item
  Finally, we examined if the judgment slopes from Hypothesis 1 would be
  predictive of recall. Hypothesis 3 examined the direct relationship of
  word relatedness on recall, while this hypothesis explored if
  participant sensitivity to word relatedness was a predictor of recall.
  For this analysis, we used a multilevel logistic regression to control
  for multiple judgment slope conditions. This hypothesis combines both
  cognitive processes into one analysis, to explore how judgment ability
  (i.e., slopes) would impact the memory process.
\end{enumerate}

\hypertarget{method}{%
\section{Method}\label{method}}

\hypertarget{participants}{%
\subsection{Participants}\label{participants}}

A power analysis was conducted using the \emph{simR} package in \emph{R}
(Green \& MacLeod, 2016). This package uses simulations to generate
power estimates for mixed linear models created from the \emph{lme4}
package in \emph{R} (Bates, Mächler, Bolker, \& Walker, 2015). The
results of this analyses suggested a minimum of 35 participants would be
required to detect an effect. However, because power often tends to be
underestimated, we extended participant recruitment as funding
permitted. In total, 112 participants took part in this study.
Participants were recruited from Amazon's Mechanical Turk, which is a
website that allows individuals to host projects and connects them with
a large pool of respondents who complete them for small amounts of money
(Buhrmester, Kwang, \& Gosling, 2011). Participant responses were
screened for a basic understanding of the study's instructions.
Responses were rejected for participants who entered related words when
numerical judgment responses were required, and for participants who
responded to the cue words during the recall phase with sentences or
phrases instead of individual words. Those that completed the study
correctly were compensated \$1.00 for their participation.

\#\#Materials

The stimuli used were sixty-three words pairs of varying associative,
semantic, and thematic relatedness which were created from the Buchanan
et al. (2013) word norm database and website. Associative relatedness
was measured with Forward Strength (FSG), which is the probability that
a cue word will elicit a desired target word (Nelson et al., 2004). This
variable ranges from zero to one wherein zero indicates no association,
while one indicates that participants would always give a target word in
response to the cue word. Semantic relatedness was measured with cosine
(COS), which is a measure of semantic feature overlap (Buchanan et al.,
2013; McRae et al., 2005; Vinson \& Vigliocco, 2008). This variable
ranges from zero to one where zero indicates no shared semantic features
between concepts and higher numbers indicate more shared features
between concepts. Thematic relatedness was calculated with Latent
Semantic Analysis (LSA), which generates a score based upon the
co-occurrences of words within a document (Landauer \& Dumais, 1997;
Landauer et al., 1998). LSA values also range from zero to one,
indicates no co-occurrence at the low end and higher co-occurrence with
higher values. These values were chosen to represent these categories
based on face validity and previous research on how word pair
psycholinguistic variables overlap (Maki \& Buchanan, 2008).

The selected stimuli included a range of values for each variable. Table
\ref{tab:stim-table} displays stimuli averages, SD, and ranges. A
complete list of stimuli can be found at \url{http://osf.io/y8h7v}. The
stimuli were arranged into three blocks for each judgment condition
described below wherein each block contained 21 word pairs. Due to
limitations of the available stimuli, blocks were structured so that
each one contained seven word pairs of low (0-.33), medium (.34-.66),
and high (.67-1.00) COS relatedness. Because of this selection process,
FSG and LSA strengths are contingent upon the selected stimuli's COS
strengths. We selected stimuli within the cosine groupings to cover a
range of FSG and LSA values, but certain combinations are often
difficult to achieve. For example, there are only four word-pairs that
are both high COS and high FSG, thus limiting the ability to manipulate
LSA. The study was built online using Qualtrics, and three surveys were
created to counter-balance the order in which judgment conditions
appeared. Each word pair appeared counter-balanced across each judgment
condition, and stimuli were randomized within each block.

\begin{table}[tbp]
\begin{center}
\begin{threeparttable}
\caption{\label{tab:stim-table}Summary Statistics for Stimuli}
\begin{tabular}{lccccccccc}
\toprule
Variable &   & COS Low &   &   & COS Average &   &   & COS High &  \\
\midrule
 & $N$ & $M$ & $SD$ & $N$ & $M$ & $SD$ & $N$ & $M$ & $SD$\\
COS & 21 & .115 & .122 & 21 & .461 & .098 & 21 & .754 & .059\\
FSG Low & 18 & .062 & .059 & 18 & .122 & .079 & 17 & .065 & .067\\
FSG Average & 3 & .413 & .093 & 2 & .411 & .046 & 2 & .505 & .175\\
FSG High & NA & NA & NA & 1 & .697 & NA & 2 & .744 & .002\\
LSA Low & 16 & .174 & .090 & 8 & .220 & .074 & 7 & .282 & .064\\
LSA Average & 5 & .487 & .126 & 10 & .450 & .111 & 12 & .478 & .095\\
LSA High & NA & NA & NA & 3 & .707 & .023 & 2 & .830 & .102\\
\bottomrule
\addlinespace
\end{tabular}
\begin{tablenotes}[para]
\textit{Note.} COS: Cosine, FSG: Forward Strength, LSA: Latent Semantic Analysis.
\end{tablenotes}
\end{threeparttable}
\end{center}
\end{table}

\hypertarget{procedure}{%
\subsection{Procedure}\label{procedure}}

The present study was divided into three phases. In the first phase,
JORs were elicited by presenting participants with word pairs and asking
them to make judgments of how related they believed the words in each
pair to be. This judgment phase consisted of three blocks of 21 word
pairs which corresponded to one of three types of word pair
relationships: associative, semantic, or thematic. Each block was
preceded by a set of instructions explaining one of the three types of
relationships, and participants were provided with examples which
illustrated the type of relationship to be judged. Participants were
then presented with the word pairs to be judged. The associative block
began by explaining associative memory and the role of free association
tasks. Participants were provided with examples of both strong and weak
associates. For example, \emph{lost} and \emph{found} and were presented
as an example of a strongly associated pair, while \emph{article} was
paired with \emph{newspaper}, \emph{the}, and \emph{clothing} to
illustrate that words can have many weak associates. The semantic
judgment block provided participants with a brief overview of how words
are related by meaning and showed examples of concepts with both high
and low feature overlap. \emph{Tortoise} and \emph{turtle} were provided
as an example of two concepts with significant overlap. Other examples
were then provided to illustrate concepts with little or no overlap. For
the thematic judgments, participants were provided with an explanation
of thematic relatedness. \emph{Tree} is explained to be related to
\emph{leaf}, \emph{fruit}, and \emph{branch}, but not \emph{computer}.
Participants were then given three concepts (\emph{lost}, \emph{old},
\emph{article}) and were asked to come up with words that they feel are
thematically related.

After viewing the examples at the start of the block, participants
completed the JOR task. Each block contained a set of instructions which
were contingent upon the type of JOR being elicited. For example,
instructions in the associative block asked participants to estimate how
many individuals out of 100 they expect would respond to the cue word
with a given target, instructions for semantic JORs asked participants
to indicate the percent of features shared between two concepts, and
instructions for the thematic JOR task asked participants to base
ratings on how likely to words would be used together in the same story.
The complete experiment can be found at \url{http://osf.io/y8h7v}, which
contains the exact instructions given to participants for each block and
displays the structure of the study. All instructions were modeled after
Buchanan (2010) and Valentine and Buchanan (2013).

In accordance with previous work on JOLs and JAM, participants made JOR
ratings using a scale of zero to one hundred, with zero indicating no
relationship, and one hundred indicating a perfect relationship.
Participants typed their responses into the survey. Once completed,
participants then completed the remaining judgment blocks in the same
manner. Each subsequent judgment block changed the type of JOR being
made. Three versions of the study were created, which counter-balanced
the order in which the judgment blocks appeared, and participants were
randomly assigned to a survey version. This resulted in each word pair
receiving a relatedness judgments on each of the three types
relationships.

After completing the judgment phase, participants were then presented
with a short distractor task to account for recency effects. In this
section, participants were presented with a randomized list of the fifty
U.S. states and were asked to arrange them in alphabetical order. This
task was timed to last two minutes. Once time had elapsed, participants
automatically progressed to the final phase, which consisted of a
cued-recall task. Participants were presented with each of the 63 cue
words from the judgment phase and were asked to complete each word pair
by responding with the correct target word. Participants were informed
that they would not be penalized for guessing. The cued-recall task
included all stimuli in a random order.

\hypertarget{results}{%
\section{Results}\label{results}}

\hypertarget{data-processing-and-descriptive-statistics}{%
\subsection{Data Processing and Descriptive
Statistics}\label{data-processing-and-descriptive-statistics}}

First, the results from the recall phase of the study was coded as zero
for incorrect responses, one for correct responses, and NA for
participants who did not complete the recall section (all or nearly all
responses were blank). All word responses to judgment items were deleted
and set to missing data. The final dataset was created by splitting the
initial data file into six sections (one for each of the three
experimental blocks and their corresponding recall scores). Each section
was individually melted using the \emph{reshape} package in \emph{R}
(Wickham, 2007) and was written as a csv file. The six output files were
then combined to form the final dataset. Code is available on our OSF
page embedded inline with the manuscript in an \emph{R} markdown
document written with the \emph{papaja} package (Aust \& Barth, 2017).
With 112 participants, the dataset in long format included 7,056 rows of
potential data (i.e., 112 participants * 63 JORs). One out of range JOR
data point (\textgreater{} 100) was corrected to NA. Missing data for
JORs or recall were then excluded from the analysis, which included word
responses to judgment items (i.e., responding with \emph{cat} instead of
a number). These items usually excluded a participant from receiving
Amazon Mechanical Turk payment, but were included in the datasets found
online. In total, 787 data points were excluded (188 JOR only, 279
recall only, 320 both), leading to a final \emph{N} of 105 participants
and 6,269 observations. Recall and JOR values were then screened for
outliers using Mahalanobis distance at \emph{p} \textless{} .001, and no
outliers were found (Tabachnick \& Fidell, 2012). To screen for
multicollinearity, we examined correlations between judgment items, COS,
LSA, and FSG. All correlations were \emph{r}s \textless{} .50.

The mean JOR for the associative condition (\emph{M} = 58.74, \emph{SD}
= 30.28) was lower than the semantic (\emph{M} = 66.98, \emph{SD} =
28.31) and thematic (\emph{M} = 71.96, \emph{SD} = 27.80) conditions.
Recall averaged over 60\% for all three conditions: associative \emph{M}
= 63.40, \emph{SD} = 48.18; semantic \emph{M} = 68.02, \emph{SD} =
46.65; thematic \emph{M} = 64.89, \emph{SD} = 47.74.

\hypertarget{hypothesis-1}{%
\subsection{Hypothesis 1}\label{hypothesis-1}}

\begin{table}[tbp]
\begin{center}
\begin{threeparttable}
\caption{\label{tab:hyp1-table1}Summary Statistics for Hypothesis 1 t-Tests}
\begin{tabular}{lccccccc}
\toprule
Variable & $M$ & $SD$ & $t$ & $df$ & $p$ & $d$ & $95\% CI$\\
\midrule
Associative Intercept & .511 & .245 & 20.864 & 99 & < .001 & 2.086 & 1.734 - 2.435\\
Associative COS & -.030 & .284 & -1.071 & 99 & .287 & -0.107 & -0.303 - 0.090\\
Associative FSG & .491 & .379 & 12.946 & 99 & < .001 & 1.295 & 1.027 - 1.559\\
Associative LSA & .035 & .317 & 1.109 & 99 & .270 & 0.111 & -0.086 - 0.307\\
Semantic Intercept & .587 & .188 & 31.530 & 101 & < .001 & 3.122 & 2.649 - 3.592\\
Semantic COS & .059 & .243 & 2.459 & 101 & .016 & 0.244 & 0.046 - 0.440\\
Semantic FSG & .118 & .382 & 3.128 & 101 & .002 & 0.310 & 0.110 - 0.508\\
Semantic LSA & .085 & .304 & 2.816 & 101 & .006 & 0.279 & 0.080 - 0.476\\
Thematic Intercept & .656 & .186 & 35.475 & 100 & < .001 & 3.530 & 3.002 - 4.048\\
Thematic COS & -.081 & .239 & -3.405 & 100 & < .001 & -0.339 & -0.539 - -0.137\\
Thematic FSG & .192 & .306 & 6.290 & 100 & < .001 & 0.626 & 0.411 - 0.838\\
Thematic LSA & .188 & .265 & 7.111 & 100 & < .001 & 0.708 & 0.488 - 0.924\\
\bottomrule
\addlinespace
\end{tabular}
\begin{tablenotes}[para]
\textit{Note.} Confidence interval for $d$ was calculated using the non-central $t$-distribution. 
\end{tablenotes}
\end{threeparttable}
\end{center}
\end{table}

Our first hypothesis sought to replicate bias and sensitivity findings
from previous research while expanding the JAM function to include
judgments based on three types of memory. FSG, COS, and LSA were used to
predict each type of relatedness judgment. JOR values were divided by
100, so as to place them on the same scale as the database norms. Slopes
and intercepts were then calculated for each participant's ratings for
each of the three JOR conditions, as long as they contained at least
nine data points out of the twenty-one that were possible. Single sample
\emph{t}-tests were then conducted to test if slope and intercept values
significantly differed from zero. See Table \ref{tab:hyp1-table1} for
means and standard deviations. Slopes were then compared to the JAM
function, which is characterized by high intercepts (between 40 and 60
on a 100 point scale) and shallow slopes (between 20 and 40). Because of
the scaling of our data, to replicate this function, we should expect to
find intercepts ranging from .40 to .60 and slopes in the range of 0.20.
to 0.40. Intercepts for associative, semantic, and thematic JORs were
each significant, and all fell within or near the expected range.
Overall, thematic JORs had the highest intercept at .656, while JORs
elicited in the associative condition had the lowest intercept at .511.

The JAM slope was successfully replicated for FSG in the associative JOR
condition, with FSG significantly predicting association, although the
slope was slightly higher than expected at .491. COS and LSA did not
significantly predict association. For semantic judgments, each of the
three database norms were significant predictors. However, JAM slopes
were not replicated for this judgment type, as FSG had the highest slope
at .118, followed by LSA .085, and then COS .059. These findings were
mirrored for thematic JORs, as each database norm was a significant
predictor, yet slopes for each predictor fell below range of the
expected JAM slopes. Again, FSG had the highest slope, this time just
out of range at .192, followed closely by LSA at .188. Interestingly,
COS slopes were found to be negative for this judgment condition, -.081.
Overall, although JAM slopes were not successfully replicated in each
JOR condition, the high intercepts and shallow slopes present across
conditions are still indicative of overconfidence and insensitivity in
participant JORs.

Additionally, we examined the frequency that each predictor variable was
the strongest predictor for each of the three JOR conditions. For the
associative condition, FSG was the strongest predictor for 64.0\% of the
participants, with COS and LSA being the strongest for only 16.0\% and
20.0\% of participants respectively. These differences were less
distinct when examining the semantic and thematic JOR conditions. In the
semantic condition, FSG was highest at 44.1\% of participants, LSA was
second at 32.4\%, and COS was least likely at 23.5\%. Finally, in the
thematic condition, LSA was most likely to be the strongest predictor
with 44.6\% of participants, with FSG being the second most likely at
36.6\%, and COS again being least likely at 18.8\%. Interestingly, in
all three conditions, COS was least likely to be the strongest
predictor, even in the semantic condition. Therefore, these results
provide evidence of the nature of judgments on the memory network as
each judgment type appeared to tap each tier differently, suggesting a
three-part system, rather than one large, encompassing memory network.

\hypertarget{hypothesis-2}{%
\subsection{Hypothesis 2}\label{hypothesis-2}}

\begin{table}[tbp]
\begin{center}
\begin{threeparttable}
\caption{\label{tab:hyp2-table}MLM Statistics for Hypothesis 2}
\small{
\begin{tabular}{lcccc}
\toprule
Variable & \multicolumn{1}{c}{$beta$} & \multicolumn{1}{c}{$SE$} & \multicolumn{1}{c}{$t$} & \multicolumn{1}{c}{$p$}\\
\midrule
Intercept & 0.603 & 0.014 & 43.287 & < .001\\
Semantic Judgments & 0.079 & 0.008 & 9.968 & < .001\\
Thematic Judgments & 0.127 & 0.008 & 16.184 & < .001\\
ZCOS & -0.103 & 0.017 & -6.081 & < .001\\
ZLSA & 0.090 & 0.022 & 4.196 & < .001\\
ZFSG & 0.271 & 0.029 & 9.420 & < .001\\
ZCOS:ZLSA & -0.141 & 0.085 & -1.650 & .099\\
ZCOS:ZFSG & -0.374 & 0.111 & -3.364 & < .001\\
ZLSA:ZFSG & -0.569 & 0.131 & -4.336 & < .001\\
ZCOS:ZLSA:ZFSG & 3.324 & 0.490 & 6.791 & < .001\\
Low COS ZLSA & 0.129 & 0.033 & 3.934 & < .001\\
Low COS ZFSG & 0.375 & 0.049 & 7.679 & < .001\\
Low COS ZLSA:ZFSG & -1.492 & 0.226 & -6.611 & < .001\\
High COS ZLSA & 0.051 & 0.031 & 1.647 & .100\\
High COS ZFSG & 0.167 & 0.034 & 4.878 & < .001\\
High COS ZLSA:ZFSG & 0.355 & 0.143 & 2.484 & .013\\
Low COS Low LSA ZFSG & 0.663 & 0.078 & 8.476 & < .001\\
Low COS High LSA ZFSG & 0.087 & 0.049 & 1.754 & .079\\
Avg COS Low LSA ZFSG & 0.381 & 0.047 & 8.099 & < .001\\
Avg COS High LSA ZFSG & 0.161 & 0.027 & 5.984 & < .001\\
High COS Low LSA ZFSG & 0.099 & 0.058 & 1.707 & .088\\
High COS High LSA ZFSG & 0.236 & 0.023 & 10.263 & < .001\\
\bottomrule
\addlinespace
\end{tabular}
}
\begin{tablenotes}[para]
\textit{Note.} Database norms were mean centered. The table shows main effects and interactions for database norms at low, average, and high levels of COS and LSA when predicting participant judgments.
\end{tablenotes}
\end{threeparttable}
\end{center}
\end{table}

\begin{figure}
\centering
\includegraphics{max_buch_JOL_files/figure-latex/hyp2graph-1.pdf}
\caption{\label{fig:hyp2graph}Simple slopes graph displaying the slope of
FSG when predicting JORs at low, average, and high LSA split by low,
average, and high COS. All variables were mean centered.}
\end{figure}

The goal of Hypothesis 2 was to test for an interaction between the
three database norms when predicting participant JORs to examine the
bidirectional network model. First, the database norms were mean
centered to control for multicollinearity. The \emph{nlme} package and
\emph{lme} function were used to calculate these analyses (Pinheiro,
Bates, Debroy, Sarkar, \& Team, 2017). A maximum likelihood multilevel
model was used to test the interaction between FSG, COS, and LSA when
predicting JOR values, with participant number used as the random
intercept factor. The type of JOR being elicited was controlled for, so
as to better assess the impact of each word overlap measure regardless
of JOR condition. Multilevel models were used to retain all data points
(rather than averaging over items and conditions) while controlling for
correlated error due to participants, which makes these models
advantageous for multiway repeated measures designs (Gelman, 2006). This
analysis resulted in a significant three-way interaction between FSG,
COS, and LSA (\(\beta\) = 3.324, \emph{p} \textless{} .001), which is
examined below in a simple slopes analysis. Table \ref{tab:hyp2-table}
includes values for main effects, two-way, and three-way interactions.

To investigate this interaction, simple slopes were calculated for low,
average, and high levels of COS. This variable was chosen for two
reasons: first, it was found to be the weakest of the three predictors
in hypothesis one, and second, manipulating COS would allow us to track
changes across FSG and LSA. Significant two-way interactions were found
between FSG and LSA at both low COS (\(\beta\) = -1.492, \emph{p}
\textless{} .001), average COS (\(\beta\) =-0.569, \emph{p} \textless{}
.001), and high COS (\(\beta\) = 0.355, \emph{p} = .013). A second level
was then added to the analysis in which simple slopes were created for
each level of LSA, allowing us to assess the effects of LSA at different
levels of COS on FSG. When both COS and LSA were low, FSG significantly
predicted JOR values (\(\beta\) = 0.663, p \textless{} .001). At low COS
and average LSA, FSG decreased but still significantly predicted JORs
(\(\beta\) = 0.375, p \textless{} .001). However, when COS was low and
LSA was high, FSG was not a significant predictor (\(\beta\) = 0.087, p
= .079). A similar set of results was found at the average COS level.
When COS was average and LSA was LOW, FSG was a significant predictor,
(\(\beta\) = 0.381, \emph{p} \textless{} .001). As LSA increased at
average COS levels, FSG decreased in strength: average COS, average LSA
FSG (\(\beta\) = 0.355, \emph{p} .013) and average COS, high LSA FSG
(\(\beta\) = 0.161, \emph{p} \textless{} .001). This finding suggests
that at low COS, LSA and FSG create a seesaw effect in which increasing
levels of thematics is counterbalanced by decreasing importance of
association when predicting JORs. FSG was not a significant predictor
when COS was high and LSA was low ( 0.099, p = .088). At high COS and
average LSA, FSG significantly predicted JORs (\(\beta\) = 0.167, p
\textless{} .001), and finally when both COS and LSA were high, FSG
increased and was a significant predictor of JOR values (\(\beta\) =
0.236, p \textless{} .001). Thus, at high levels of semantic overlap,
associative and thematic overlap are complementary when predicting JOR
ratings, increasing together as semantic strength increases. Figure
\ref{fig:hyp2graph} displays the three-way interaction wherein the top
row of figures indicates the seesaw effect, as thematic strength
increases, the predictive ability of associative overlap decreases in
strength. The bottom row indicates the complementary effect where
increases in LSA occur with increases in FSG predictor strength.
Therefore, the cognitive process of judgment appears to be interactive
in nature across these three types of memory information.

\hypertarget{hypothesis-3}{%
\subsection{Hypothesis 3}\label{hypothesis-3}}

\begin{table}[tbp]
\begin{center}
\begin{threeparttable}
\caption{\label{tab:hyp3-table}MLM Statistics for Hypothesis 3}
\small{
\begin{tabular}{lcccc}
\toprule
Variable & \multicolumn{1}{c}{$beta$} & \multicolumn{1}{c}{$SE$} & \multicolumn{1}{c}{$z$} & \multicolumn{1}{c}{$p$}\\
\midrule
Intercept & 0.301 & 0.138 & 2.188 & .029\\
Semantic Judgments & 0.201 & 0.074 & 2.702 & .007\\
Thematic Judgments & -0.001 & 0.075 & -0.020 & .984\\
Judged Values & 0.686 & 0.115 & 5.956 & < .001\\
ZCOS & 0.594 & 0.179 & 3.320 & < .001\\
ZLSA & -0.350 & 0.204 & -1.714 & .087\\
ZFSG & 3.085 & 0.302 & 10.205 & < .001\\
ZCOS:ZLSA & 2.098 & 0.837 & 2.506 & .012\\
ZCOS:ZFSG & 1.742 & 1.306 & 1.334 & .182\\
ZLSA:ZFSG & -1.017 & 1.484 & -0.685 & .493\\
ZCOS:ZLSA:ZFSG & 24.572 & 6.048 & 4.063 & < .001\\
Low COS ZLSA & -0.933 & 0.301 & -3.099 & .002\\
Low COS ZFSG & 2.601 & 0.471 & 5.521 & < .001\\
Low COS ZLSA:ZFSG & -7.845 & 2.204 & -3.560 & < .001\\
High COS ZLSA & 0.233 & 0.317 & 0.737 & .461\\
High COS ZFSG & 3.569 & 0.470 & 7.586 & < .001\\
High COS ZLSA:ZFSG & 5.811 & 2.231 & 2.605 & .009\\
Low COS Low LSA ZFSG & 4.116 & 0.741 & 5.558 & < .001\\
Low COS High LSA ZFSG & 1.086 & 0.501 & 2.166 & .030\\
High COS Low LSA ZFSG & 2.447 & 0.811 & 3.018 & .003\\
High COS High LSA ZFSG & 4.692 & 0.388 & 12.083 & < .001\\
\bottomrule
\addlinespace
\end{tabular}
}
\begin{tablenotes}[para]
\textit{Note.} Database norms were mean centered. The table shows main effects and interactions for database norms at low, average, and high levels of COS and LSA when predicting recall.
\end{tablenotes}
\end{threeparttable}
\end{center}
\end{table}

\begin{figure}
\centering
\includegraphics{max_buch_JOL_files/figure-latex/hyp3graph-1.pdf}
\caption{\label{fig:hyp3graph}Simple slopes graph displaying the slope of
FSG when predicting recall at low, average, and high LSA split by low,
average, and high COS. All variables were mean centered.}
\end{figure}

Given the results of Hypothesis 2, we then sought to extend the analysis
to participant recall scores. A multilevel logistic regression was used
with the \emph{lme4} package and \emph{glmer()} function (Pinheiro et
al., 2017), testing the interaction between FSG, COS, and LSA when
predicting participant recall. As with the previous hypothesis, we
controlled for JOR condition and, additionally, covaried JOR ratings.
Participants were used as a random intercept factor. Judged values were
a significant predictor of recall, (\(\beta\) = 0.686, \emph{p}
\textless{} .001) where increases in judged strength predicted increases
in recall. A significant three-way interaction was detected between FSG,
COS, and LSA (\(\beta\) = 24.572, \emph{p} \textless{} .001). See Table
\ref{tab:hyp3-table} for main effects, two-way, and three-way
interaction values.

The same moderation process used in Hypothesis 2 was then repeated, with
simple slopes first calculated at low, average, and high levels of COS.
This set of analyses resulted in significant two-way interactions
between LSA and FSG at low COS (\(\beta\) = -7.845, \emph{p} \textless{}
.001) and high COS (\(\beta\) = 5.811, \emph{p} = .009). No significant
two-way interaction was found at average COS (\(\beta\) = -1.017,
\emph{p} = .493). Following the design of hypothesis two, simple slopes
were then calculated for low, average, and high levels of LSA at the low
and high levels of COS, allowing us to assess how FSG effects recall at
varying levels of both COS and LSA. When both COS and LSA were low, FSG
was a significant predictor of recall (\(\beta\) = 4.116, \emph{p}
\textless{} .001). At low COS and average LSA, FSG decreased from both
low levels, but was still a significant predictor (\(\beta\) = 2.601,
\emph{p} \textless{} .001), and finally, low COS and high LSA, FSG was
the weakest predictor of the three (\(\beta\) = 1.086, \emph{p} = .030).
As with Hypothesis 2, LSA and FSG counterbalanced one another, wherein
the increasing levels of thematics led to a decrease in the importance
of association in predicting recall. At high COS and low LSA, FSG was a
significant predictor (\(\beta\) = 2.447, \emph{p} = .003). When COS was
high and LSA was average, FSG increased as a predictor and remained
significant (\(\beta\) = 3.569, \emph{p} \textless{} .001). This finding
repeated when both COS and LSA were high, with FSG increasing as a
predictor of recall (\(\beta\) = 4.692, \emph{p} \textless{} .001).
Therefore, at high levels of at high levels of semantics, thematics and
association are complementary predictors of recall, increasing together
and extending the findings of Hypothesis 2 to participant recall. Figure
\ref{fig:hyp3graph} displays the three-way interaction. The top left
figure indicates the counterbalancing effect of recall of LSA and FSG,
while the top right figure shows no differences in simple slopes for
average levels of cosine. The bottom left figure indicates the
complementary effects where LSA and FSG increase together as predictors
of recall at high COS levels.

\hypertarget{hypothesis-4}{%
\subsection{Hypothesis 4}\label{hypothesis-4}}

\begin{table}[tbp]
\begin{center}
\begin{threeparttable}
\caption{\label{tab:hyp4-table}MLM Statistics for Hypothesis 4}
\begin{tabular}{lcccc}
\toprule
Variable & \multicolumn{1}{c}{$b$} & \multicolumn{1}{c}{$SE$} & \multicolumn{1}{c}{$z$} & \multicolumn{1}{c}{$p$}\\
\midrule
(Intercept) & -0.432 & 0.439 & -0.983 & .326\\
ACOS & 0.314 & 0.550 & 0.572 & .568\\
ALSA & 0.501 & 0.463 & 1.081 & .279\\
AFSG & 0.898 & 0.337 & 2.667 & .008\\
AIntercept & 1.514 & 0.604 & 2.507 & .012\\
(Intercept) & -0.827 & 0.463 & -1.787 & .074\\
SCOS & 2.039 & 0.518 & 3.939 & < .001\\
SLSA & 1.061 & 0.455 & 2.335 & .020\\
SFSG & 0.381 & 0.289 & 1.319 & .187\\
SIntercept & 2.292 & 0.681 & 3.363 & < .001\\
(Intercept) & 0.060 & 0.599 & 0.101 & .920\\
TCOS & 0.792 & 0.566 & 1.401 & .161\\
TLSA & 0.896 & 0.529 & 1.694 & .090\\
TFSG & -0.394 & 0.441 & -0.894 & .371\\
TIntercept & 1.028 & 0.756 & 1.360 & .174\\
\bottomrule
\addlinespace
\end{tabular}
\begin{tablenotes}[para]
\textit{Note.} Each judgment-database bias and sensitivity predicting recall for corresponding judgment block. A: Associative, S: Semantic, T: Thematic.
\end{tablenotes}
\end{threeparttable}
\end{center}
\end{table}

In our fourth and final hypothesis, we investigated whether the JOR
slopes and intercepts obtained in Hypothesis 1 would be predictive of
recall ability. Whereas Hypothesis 3 indicated that word relatedness was
directly related to recall performance, this hypothesis instead looked
at whether or not participants' sensitivity and bias to word relatedness
could be used a predictor of recall (Maki, 2007b). This analysis was
conducted with a multilevel logistic regression, as described in
Hypothesis 3, where each database slope and intercept was used as
predictors of recall using participant as a random intercept factor.
These analyses were separated by judgment condition, so that each set of
JOR slopes and intercepts were used to predict recall. The separation
controlled for the number of variables in the equation, as all slopes
and intercepts would have resulted in overfitting. These values were
obtained from Hypothesis 1 where each participant's individual slopes
and intercepts were calculated for associative, semantic, and thematic
JOR conditions. Table \ref{tab:hyp1-table1} shows average slopes and
intercepts for recall for each of the three types of memory, and Table
\ref{tab:hyp4-table} portrays the regression coefficients and
statistics. In the associative condition, FSG slope significantly
predicted recall (\emph{b} = 0.898, \emph{p} = .008), while COS slope
(\emph{b} = 0.314, \emph{p} = .568) and LSA slope (\emph{b} = 0.501,
\emph{p} = .279) were non-significant. In the semantic condition, COS
slope (\emph{b} = 2.039, \emph{p} \textless{} .001) and LSA slope
(\emph{b} = 1.061, \emph{p} = .020) were both found to be significant
predictors of recall. FSG slope was non-significant in this condition
(\emph{b} = 0.381, \emph{p} = .187). Finally, no predictors were
significant in the thematic condition, though LSA slope was found to be
the strongest (\emph{b} = 0.896, \emph{p} = .090). This analysis
indicated the extent to which the cognitive processes are related to
each other as part of the memory network (i.e., judgment sensitivity
predicting recall), furthering Hypothesis 2 and 3 which illustrated the
nature of those cognitive processes' relationship with the underlying
memory network.

\hypertarget{discussion}{%
\section{Discussion}\label{discussion}}

This study investigated the relationship between associative, semantic,
and thematic word relations and their effect on participant JORs and
recall performance through the testing of four hypotheses. In our first
hypothesis, bias and sensitivity findings first proposed by Maki (2007a)
were successfully replicated in the associative condition, with slope
and intercept values falling within the expected range. While these
findings were not fully replicated when extending the analysis to
include semantic and thematic JORs (as slopes in these conditions did
not fall within the appropriate range), participants still displayed
high intercepts and shallow slopes, suggesting overconfidence in
judgment making and an insensitivity to changes in strength between
pairs. Additionally, when looking at the frequency that each predictor
was the strongest in making JORs, FSG was the best predictor for both
the associative and semantic conditions, while LSA was the best
predictor in the thematic condition. In each of the three conditions,
COS was the weakest predictor, even when participants were asked to make
semantic judgments. This finding suggests that associative relationships
seem to take precedence over semantic relationships when judging pair
relatedness, regardless of what type of JOR is being elicited.
Additionally, this finding may be taken as further evidence of a
separation between associative information and semantic information, in
which associative information is always processed, while semantic
information may be suppressed due to task demands (Buchanan, 2010;
Hutchison \& Bosco, 2007).

Our second hypothesis examined the three-way interaction between FSG,
COS, and LSA when predicting participant JORs. At low semantic overlap,
a seesaw effect was found in which increases in thematic strength led to
decreases in associative predictiveness. This finding was then
replicated in Hypothesis 3 when extending the analysis to predict
recall. By limiting the semantic relationships between pairs, an
increased importance is placed on the role of associations and thematics
when making relatedness judgments or retrieving pairs. In such cases,
increasing the amount of thematic overlap between pairs results in
thematic relationships taking precedent over associative relationships.
However, when semantic overlap was high, a complementary relationship
was found in which increases in thematic strength in turn led to
increases in the strength of FSG as a predictor. This result suggests
that at high semantic overlap, associations and thematic relations build
upon one another. Because thematics is tied to both semantic overlap and
item associations, the presence of strong thematic relationships between
pairs during conditions of high semantic overlap boosts the predictive
ability of associative word norms for both recall and JORs.

Finally, our fourth hypothesis used the JOR slopes and intercepts
calculated in Hypothesis 1 to investigate if participants' bias and
sensitivity to word relatedness could be used to predict recall. For the
associative condition, the FSG slope significantly predicted recall. In
the semantic condition, recall was significantly predicted by both the
COS and LSA slopes, with COS being the strongest. However, for the
thematic condition, although the LSA slope was the strongest, no
predictors were significant. One explanation for this finding is that
thematic relationships between item pairs act as a blend between
associations and semantics. As such, LSA faces increased competition
from the associative and semantic database norms when predicting recall
in this manner. Additionally, the dominance of FSG when predicting
recall in the associative condition may be attributed to word
associations being more accessible (and, thus, easier to process) than
semantic or thematic relations between pairs.

Overall, our findings indicated the degree to which the processing of
associative, semantic, and thematic information impacts retrieval and
judgment making tasks and the interactive relationship that exists
between these three types of lexical information. While previous
research has shown that memory networks are divided into separate
systems which handle storage and processing for meaning and association
(see Ferrand \& New, 2004 for a review), the presence of these
interactions suggests that connections exist between these individual
memory networks, linking them to one another. As such, we suggest that
these memory systems may be connected in such a way to form a
three-tiered, interconnected system. First, information enters the
semantic memory network, which processes features of concepts and
provides a means of categorizing items based on the similarity of their
features. Next, the associative network adds information for items based
on contexts generated by reading or speech. Finally, the thematic
network pulls in information from both the semantic and associative
networks to create a mental representation of both the item and its
place in the world relative to other concepts. This study did not
explore the timing of information input from each of these systems, but
it may be similar to a dual-route model of reading and naming, in that
each runs in parallel contributing the judgment and recall process
(Coltheart, Curtis, Atkins, \& Haller, 1993).

Viewing this model purely through the lens of semantic memory, it draws
comparison to dynamic attractor models (Hopfield, 1982; Jones et al.,
2015; McLeod et al., 2000). One of the defining features of dynamic
attractor models is that they allow for some type of bidirectionally or
feedback between connections in the network. In the study of semantic
memory, these models are useful for taking into account multiple
restraints such as links between semantics and the orthography of the
concept in question. Our hypothesis extends this notion as a means of
framing how these three memory systems are connected. The underlying
meaning of a concept is linked with both information pertaining to its
co-occurrences in everyday language and information relating to the
general contexts in which it typically appears.

How then does this hypothesis lend itself towards the broader context of
psycholinguistic research? One application of this hypothesis may be
models of word recognition. One popular class of models are those based
upon Seidenberg and McClelland (1989) \enquote{triangle model} (see
Harley, 2008 for a review). They key feature of these models is that
they recognize speech and reading based upon the orthography, phonology,
and meaning of words in a bidirectional manner, similar to the models
described above. Harm and Seidenberg (2004) developed a version which
included a focus on semantics, with word meaning being related to input
from the orthography and phonology components of the model. Our findings
from the present study further suggest that thematic and associative
knowledge is incorporated with meaning. One way of framing our results
within this literature is to consider the semantic section of the
triangle model as being comprised of these three tiers, and that concept
information is processed to some degree on each of these domains. One
area for future studies of this nature may be investigating how aspects
of orthography and phonology impact these memory networks. Additionally,
future studies may wish to consider elements of thematic and associative
knowledge when examining semantic based tasks, such as word recognition
and reading, as thematic and associative information is interconnected
with the semantic network. Ultimately, further studies will be needed to
fully understand the interconnections between the semantic, thematic,
and associative networks.

\newpage

\hypertarget{references}{%
\section{References}\label{references}}

\setlength{\parindent}{-0.5in}
\setlength{\leftskip}{0.5in}

\hypertarget{refs}{}
\leavevmode\hypertarget{ref-Aust2017}{}%
Aust, F., \& Barth, M. (2017). papaja: Create APA manuscripts with R
Markdown. Retrieved from \url{https://github.com/crsh/papaja}

\leavevmode\hypertarget{ref-Bates2015}{}%
Bates, D., Mächler, M., Bolker, B., \& Walker, S. (2015). Fitting linear
mixed-effects models using lme4. \emph{Journal of Statistical Software},
\emph{67}(1), 1--48.
doi:\href{https://doi.org/10.18637/jss.v067.i01}{10.18637/jss.v067.i01}

\leavevmode\hypertarget{ref-Buchanan2010}{}%
Buchanan, E. M. (2010). Access into memory: Differences in judgments and
priming for semantic and associative memory. \emph{Journal of Scientific
Psychology}, \emph{March}, 1--8.

\leavevmode\hypertarget{ref-Buchanan2013}{}%
Buchanan, E. M., Holmes, J. L., Teasley, M. L., \& Hutchison, K. A.
(2013). English semantic word-pair norms and a searchable Web portal for
experimental stimulus creation. \emph{Behavior Research Methods},
\emph{45}(3), 746--757.
doi:\href{https://doi.org/10.3758/s13428-012-0284-z}{10.3758/s13428-012-0284-z}

\leavevmode\hypertarget{ref-Buhrmester2011}{}%
Buhrmester, M., Kwang, T., \& Gosling, S. D. (2011). Amazon's Mechanical
Turk. \emph{Perspectives on Psychological Science}, \emph{6}(1), 3--5.
doi:\href{https://doi.org/10.1177/1745691610393980}{10.1177/1745691610393980}

\leavevmode\hypertarget{ref-Chow2014}{}%
Chow, B. W.-Y. (2014). The differential roles of paired associate
learning in Chinese and English word reading abilities in bilingual
children. \emph{Reading and Writing}, \emph{27}(9), 1657--1672.
doi:\href{https://doi.org/10.1007/s11145-014-9514-3}{10.1007/s11145-014-9514-3}

\leavevmode\hypertarget{ref-Coltheart1993}{}%
Coltheart, M., Curtis, B., Atkins, P., \& Haller, M. (1993). Models of
reading aloud: Dual-route and parallel-distributed-processing
approaches. \emph{Psychological Review}, \emph{100}(4), 589--608.
doi:\href{https://doi.org/10.1037/0033-295X.100.4.589}{10.1037/0033-295X.100.4.589}

\leavevmode\hypertarget{ref-DeDeyne2018}{}%
De Deyne, S., Navarro, D. J., Perfors, A., Brysbaert, M., \& Storms, G.
(2018). \emph{Measuring the associative structure of English: The
``Small World of Words'' norms for word association}.

\leavevmode\hypertarget{ref-DeDeyne2013}{}%
De Deyne, S., Navarro, D. J., \& Storms, G. (2013). Better explanations
of lexical and semantic cognition using networks derived from continued
rather than single-word associations. \emph{Behavior Research Methods},
\emph{45}(2), 480--498.
doi:\href{https://doi.org/10.3758/s13428-012-0260-7}{10.3758/s13428-012-0260-7}

\leavevmode\hypertarget{ref-Ferrand2004}{}%
Ferrand, L., \& New, B. (2004). Semantic and associative priming in the
mental lexicon. In P. Bonin (Ed.), \emph{The mental lexicon} (pp.
25--43). Hauppauge, NY: Nova Science.

\leavevmode\hypertarget{ref-Gelman2006}{}%
Gelman, A. (2006). Multilevel (hierarchical) modeling: What it can and
cannot do. \emph{Technometrics}, \emph{48}(3), 432--435.
doi:\href{https://doi.org/10.1198/004017005000000661}{10.1198/004017005000000661}

\leavevmode\hypertarget{ref-Green2016}{}%
Green, P., \& MacLeod, C. J. (2016). SIMR: an R package for power
analysis of generalized linear mixed models by simulation. \emph{Methods
in Ecology and Evolution}, \emph{7}(4), 493--498.
doi:\href{https://doi.org/10.1111/2041-210X.12504}{10.1111/2041-210X.12504}

\leavevmode\hypertarget{ref-Harley2008}{}%
Harley, T. (2008). \emph{The psychology of language: From data to
theory} (Third.). New York: Psychology Press.

\leavevmode\hypertarget{ref-Harm2004}{}%
Harm, M. W., \& Seidenberg, M. S. (2004). Computing the meanings of
words in reading: Cooperative division of labor between visual and
phonological processes. \emph{Psychological Review}, \emph{111}(3),
662--720.
doi:\href{https://doi.org/10.1037/0033-295X.111.3.662}{10.1037/0033-295X.111.3.662}

\leavevmode\hypertarget{ref-Hertzog2002}{}%
Hertzog, C., Kidder, D. P., Powell-Moman, A., \& Dunlosky, J. (2002).
Aging and monitoring associative learning: Is monitoring accuracy spared
or impaired? \emph{Psychology and Aging}, \emph{17}(2), 209--225.
doi:\href{https://doi.org/10.1037/0882-7974.17.2.209}{10.1037/0882-7974.17.2.209}

\leavevmode\hypertarget{ref-Hopfield1982}{}%
Hopfield, J. J. (1982). Neural networks and physical systems with
emergent collective computational abilities. \emph{Proceedings of the
National Academy of Sciences}, \emph{79}(8), 2554--2558.
doi:\href{https://doi.org/10.1073/pnas.79.8.2554}{10.1073/pnas.79.8.2554}

\leavevmode\hypertarget{ref-Hutchison2003}{}%
Hutchison, K. A. (2003). Is semantic priming due to association strength
or feature overlap? A microanalytic review. \emph{Psychonomic Bulletin
\& Review}, \emph{10}(4), 785--813.
doi:\href{https://doi.org/10.3758/BF03196544}{10.3758/BF03196544}

\leavevmode\hypertarget{ref-Hutchison2007}{}%
Hutchison, K. A., \& Bosco, F. A. (2007). Congruency effects in the
letter search task: Semantic activation in the absence of priming.
\emph{Memory \& Cognition}, \emph{35}(3), 514--525.
doi:\href{https://doi.org/10.3758/BF03193291}{10.3758/BF03193291}

\leavevmode\hypertarget{ref-Jiang1997}{}%
Jiang, J. J., \& Conrath, D. W. (1997). Semantic similarity based on
corpus statistics and lexical taxonomy. \emph{Proceedings of
International Conference Research on Computational Linguistics (ROCLING
X)}. Retrieved from \url{http://arxiv.org/abs/cmp-lg/9709008}

\leavevmode\hypertarget{ref-Jones2012}{}%
Jones, L. L., \& Golonka, S. (2012). Different influences on lexical
priming for integrative, thematic, and taxonomic relations.
\emph{Frontiers in Human Neuroscience}, \emph{6}(July), 1--17.
doi:\href{https://doi.org/10.3389/fnhum.2012.00205}{10.3389/fnhum.2012.00205}

\leavevmode\hypertarget{ref-Jones2015}{}%
Jones, M. N., Willits, J., \& Dennis, S. (2015). Models of semantic
memory. In A. T. Townsend \& J. R. Busemeyer (Eds.), \emph{Oxford
handbook of mathematical and computational psychology} (pp. 232--254).
Oxford University Press.
doi:\href{https://doi.org/10.1093/oxfordhb/9780199957996.013.11}{10.1093/oxfordhb/9780199957996.013.11}

\leavevmode\hypertarget{ref-Jouravlev2016}{}%
Jouravlev, O., \& McRae, K. (2016). Thematic relatedness production
norms for 100 object concepts. \emph{Behavior Research Methods},
\emph{48}(4), 1349--1357.
doi:\href{https://doi.org/10.3758/s13428-015-0679-8}{10.3758/s13428-015-0679-8}

\leavevmode\hypertarget{ref-Koriat2005}{}%
Koriat, A., \& Bjork, R. A. (2005). Illusions of competence in
monitoring one's knowledge during study. \emph{Journal of Experimental
Psychology: Learning, Memory, and Cognition}, \emph{31}(2), 187--194.
doi:\href{https://doi.org/10.1037/0278-7393.31.2.187}{10.1037/0278-7393.31.2.187}

\leavevmode\hypertarget{ref-Landauer1997}{}%
Landauer, T. K., \& Dumais, S. T. (1997). A solution to Plato's problem:
The latent semantic analysis theory of acquisition, induction, and
representation of knowledge. \emph{Psychological Review}, \emph{104}(2),
211--240.
doi:\href{https://doi.org/10.1037//0033-295X.104.2.211}{10.1037//0033-295X.104.2.211}

\leavevmode\hypertarget{ref-Landauer1998}{}%
Landauer, T. K., Foltz, P. W., Laham, D., Folt, P. W., \& Laham, D.
(1998). An introduction to latent semantic analysis. \emph{Discourse
Processes}, \emph{25}(2), 259--284.
doi:\href{https://doi.org/10.1080/01638539809545028}{10.1080/01638539809545028}

\leavevmode\hypertarget{ref-Lucas2000}{}%
Lucas, M. (2000). Semantic priming without association: a meta-analytic
review. \emph{Psychonomic Bulletin \& Review}, \emph{7}(4), 618--630.
doi:\href{https://doi.org/10.3758/BF03212999}{10.3758/BF03212999}

\leavevmode\hypertarget{ref-Lund1996}{}%
Lund, K., \& Burgess, C. (1996). Producing high-dimensional semantic
spaces from lexical co-occurrence. \emph{Behavior Research Methods,
Instruments, \& Computers}, \emph{28}(2), 203--208.
doi:\href{https://doi.org/10.3758/BF03204766}{10.3758/BF03204766}

\leavevmode\hypertarget{ref-Maki2007a}{}%
Maki, W. S. (2007a). Judgments of associative memory. \emph{Cognitive
Psychology}, \emph{54}(4), 319--353.
doi:\href{https://doi.org/10.1016/j.cogpsych.2006.08.002}{10.1016/j.cogpsych.2006.08.002}

\leavevmode\hypertarget{ref-Maki2007}{}%
Maki, W. S. (2007b). Separating bias and sensitivity in judgments of
associative memory. \emph{Journal of Experimental Psychology. Learning,
Memory, and Cognition}, \emph{33}(1), 231--237.
doi:\href{https://doi.org/10.1037/0278-7393.33.1.231}{10.1037/0278-7393.33.1.231}

\leavevmode\hypertarget{ref-Maki2008}{}%
Maki, W. S., \& Buchanan, E. M. (2008). Latent structure in measures of
associative, semantic, and thematic knowledge. \emph{Psychonomic
Bulletin \& Review}, \emph{15}(3), 598--603.
doi:\href{https://doi.org/10.3758/PBR.15.3.598}{10.3758/PBR.15.3.598}

\leavevmode\hypertarget{ref-Maki2004}{}%
Maki, W. S., McKinley, L. N., \& Thompson, A. G. (2004). Semantic
distance norms computed from an electronic dictionary (WordNet).
\emph{Behavior Research Methods, Instruments, \& Computers},
\emph{36}(3), 421--431.
doi:\href{https://doi.org/10.3758/BF03195590}{10.3758/BF03195590}

\leavevmode\hypertarget{ref-McLeod2000}{}%
McLeod, P., Shallice, T., \& Plaut, D. C. (2000). Attractor dynamics in
word recognition: converging evidence from errors by normal subjects,
dyslexic patients and a connectionist model. \emph{Cognition},
\emph{74}(1), 91--114.
doi:\href{https://doi.org/10.1016/S0010-0277(99)00067-0}{10.1016/S0010-0277(99)00067-0}

\leavevmode\hypertarget{ref-McRae2005}{}%
McRae, K., Cree, G. S., Seidenberg, M. S., \& McNorgan, C. (2005).
Semantic feature production norms for a large set of living and
nonliving things. \emph{Behavior Research Methods}, \emph{37}(4),
547--559.
doi:\href{https://doi.org/10.3758/BF03192726}{10.3758/BF03192726}

\leavevmode\hypertarget{ref-Meyer1971}{}%
Meyer, D. E., \& Schvaneveldt, R. W. (1971). Facilitation in recognizing
pairs of words: Evidence of a dependence between retrieval operations.
\emph{Journal of Experimental Psychology}, \emph{90}(2), 227--234.
doi:\href{https://doi.org/10.1037/h0031564}{10.1037/h0031564}

\leavevmode\hypertarget{ref-Meyer1975}{}%
Meyer, D. E., Schvaneveldt, R. W., \& Ruddy, M. G. (1975). Loci of
contextual effects on visual word-recognition. In P. M. A. Rabbitt
(Ed.), \emph{Attention and performance v}. London, UK: Academic Press.

\leavevmode\hypertarget{ref-Miller1995}{}%
Miller, G. A. (1995). WordNet: a lexical database for English.
\emph{Communications of the ACM}, \emph{38}(11), 39--41.
doi:\href{https://doi.org/10.1145/219717.219748}{10.1145/219717.219748}

\leavevmode\hypertarget{ref-Nelson2000}{}%
Nelson, D. L., McEvoy, C. L., \& Dennis, S. (2000). What is free
association and what does it measure? \emph{Memory \& Cognition},
\emph{28}(6), 887--899.
doi:\href{https://doi.org/10.3758/BF03209337}{10.3758/BF03209337}

\leavevmode\hypertarget{ref-Nelson2004}{}%
Nelson, D. L., McEvoy, C. L., \& Schreiber, T. A. (2004). The University
of South Florida free association, rhyme, and word fragment norms.
\emph{Behavior Research Methods, Instruments, \& Computers},
\emph{36}(3), 402--407.
doi:\href{https://doi.org/10.3758/BF03195588}{10.3758/BF03195588}

\leavevmode\hypertarget{ref-Paivio1969}{}%
Paivio, A. (1969). Mental imagery in associative learning and memory.
\emph{Psychological Review}, \emph{76}(3), 241--263.
doi:\href{https://doi.org/10.1037/h0027272}{10.1037/h0027272}

\leavevmode\hypertarget{ref-Pinheiro2017}{}%
Pinheiro, J., Bates, D., Debroy, S., Sarkar, D., \& Team, R. C. (2017).
nlme: Linear and nonlinear mixed effects models. Retrieved from
\url{https://cran.r-project.org/package=nlme}

\leavevmode\hypertarget{ref-Plaut1995}{}%
Plaut, D. C. (1995). Semantic and associative priming in a distributed
attractor network. \emph{Proceedings of the 17th Annual Conference of
the Cognitive Science Society}, 37--42.

\leavevmode\hypertarget{ref-Richardson1998}{}%
Richardson, J. T. E. (1998). The availability and effectiveness of
reported mediators in associative learning: A historical review and an
experimental investigation. \emph{Psychonomic Bulletin \& Review},
\emph{5}(4), 597--614.
doi:\href{https://doi.org/10.3758/BF03208837}{10.3758/BF03208837}

\leavevmode\hypertarget{ref-Riordan2011}{}%
Riordan, B., \& Jones, M. N. (2011). Redundancy in perceptual and
linguistic experience: Comparing feature-based and distributional models
of semantic representation. \emph{Topics in Cognitive Science},
\emph{3}(2), 303--345.
doi:\href{https://doi.org/10.1111/j.1756-8765.2010.01111.x}{10.1111/j.1756-8765.2010.01111.x}

\leavevmode\hypertarget{ref-Rogers2006}{}%
Rogers, T. T., \& McClelland, J. L. (2006). \emph{Semantic cognition}.
Cambridge, MA: MIT Press.

\leavevmode\hypertarget{ref-Rumelhart1986}{}%
Rumelhart, D. E., McClelland, J. L., \& PDP Research Group. (1986).
\emph{Parallel distributed processing: Explorations in the
microstructure of cognition. Volume 1}. Cambridge, MA: MIT Press.

\leavevmode\hypertarget{ref-Schwartz2013}{}%
Schwartz, B. L., \& Brothers, B. R. (2013). Survival processing does not
improve paired-associate learning. In B. L. Schwartz, M. L. Howe, M. P.
Toglia, \& H. Otgaar (Eds.), \emph{What is adaptive about adaptive
memory?} (pp. 159--171). Oxford University Press.
doi:\href{https://doi.org/10.1093/acprof:oso/9780199928057.003.0009}{10.1093/acprof:oso/9780199928057.003.0009}

\leavevmode\hypertarget{ref-Seidenberg1989a}{}%
Seidenberg, M. S., \& McClelland, J. L. (1989). A distributed,
developmental model of word recognition and naming. \emph{Psychological
Review}, \emph{96}(4), 523--568.
doi:\href{https://doi.org/10.1037//0033-295X.96.4.523}{10.1037//0033-295X.96.4.523}

\leavevmode\hypertarget{ref-Smythe1968}{}%
Smythe, P. C., \& Paivio, A. (1968). A comparison of the effectiveness
of word Imagery and meaningfulness in paired-associate learning of
nouns. \emph{Psychonomic Science}, \emph{10}(2), 49--50.
doi:\href{https://doi.org/10.3758/BF03331401}{10.3758/BF03331401}

\leavevmode\hypertarget{ref-Tabachnick2012}{}%
Tabachnick, B. G., \& Fidell, L. S. (2012). \emph{Using multivariate
statistics} (Sixth.). Boston, MA: Pearson.

\leavevmode\hypertarget{ref-Valentine2013}{}%
Valentine, K. D., \& Buchanan, E. M. (2013). JAM-boree: An application
of observation oriented modelling to judgements of associative memory.
\emph{Journal of Cognitive Psychology}, \emph{25}(4), 400--422.
doi:\href{https://doi.org/10.1080/20445911.2013.775120}{10.1080/20445911.2013.775120}

\leavevmode\hypertarget{ref-Vinson2008}{}%
Vinson, D. P., \& Vigliocco, G. (2008). Semantic feature production
norms for a large set of objects and events. \emph{Behavior Research
Methods}, \emph{40}(1), 183--190.
doi:\href{https://doi.org/10.3758/BRM.40.1.183}{10.3758/BRM.40.1.183}

\leavevmode\hypertarget{ref-Wickham2007}{}%
Wickham, H. (2007). Reshaping data with the reshape package.
\emph{Journal of Statistical Software}, \emph{21}(12), 1--20.
doi:\href{https://doi.org/10.18637/jss.v021.i12}{10.18637/jss.v021.i12}


\end{document}
